%
% A compiler avec xelatex !
% xelatex -output-driver="xdvipdfmx" deepmath_cover1.tex
%
\documentclass[tikz,border=0mm]{standalone}
\usetikzlibrary{calc}
% =====================================================
% For the Python logo (that was in SVG)
% =====================================================
\usetikzlibrary{svg.path}
% =====================================================
% For the Hilbert curve
% =====================================================
\usetikzlibrary{shadings}

% =====================================================
% The font
% =====================================================
\usepackage{fontspec}
\newfontfamily\titlefont{AlteDin1451Mittelschrift.ttf}
\newfontfamily\roboto{Roboto}
\setmainfont{Roboto}
%\newfontfamily\authorfont{Times New Roman}
%\newfontfamily\authorfont{Roboto}
\newfontfamily\authorfont{AlteDin1451Mittelschrift.ttf}

% =====================================================
% The colors
% =====================================================
\definecolor{scratchgreen}{rgb}{.43000793,.78393555,.27934265}
\definecolor{scratchviolet}{rgb}{.50601196,.10708618,.55937195}

\definecolor{goldleaf}{HTML}{D1B280}
% 16 Day & Night
\definecolor{darknavy}{HTML}{011A27}
\definecolor{blueberry}{HTML}{073852}
\definecolor{tangerine}{HTML}{F0810F}
\definecolor{daffodil}{HTML}{E7DF44}

% Arnaud
\definecolor{hazelnut}{HTML}{B99A77}
\definecolor{aquablue}{HTML}{31A9B9}
\definecolor{tomato}{HTML}{CF3721}
\definecolor{avocado}{HTML}{258039}

\definecolor{cayenne}{HTML}{AF4425}
\definecolor{cinnamon}{HTML}{662E1C}
\definecolor{caramel}{HTML}{C9A66B}

% Style marron
%\definecolor{coulfond}{HTML}{DDC5A2}
%\definecolor{coulup}{HTML}{301B28}
%\definecolor{couldown}{HTML}{B6452C}
%\definecolor{coulreste}{HTML}{523634}

% Couleur épreuve
\definecolor{coulfond}{HTML}{CCCCCC}
\definecolor{coulup}{HTML}{D01A55}
\definecolor{couldown}{HTML}{C1D208}
\definecolor{coulreste}{HTML}{666666}

% Colors by function
\colorlet{background}{coulfond}
\tikzset{
  hilbert/.style = {bottom color=black, top color=white, draw=black, opacity=.14},
  spine/.style = {fill, opacity=.1}
}

\colorlet{author}{black}

\colorlet{pythonlogoup}{coulup}
\colorlet{pythonlogodown}{couldown}
\colorlet{titlepython}{white}
\colorlet{titleau}{coulreste}
\colorlet{titlelycee}{white}

\colorlet{subtitlealgo}{coulreste}

\colorlet{spinepython}{coulup}
\colorlet{spineau}{couldown}
\colorlet{spinelycee}{coulup}
\colorlet{spinealgo}{white}


%=======================================================
% The cover size
%=======================================================
% Calculate the splin
\def\numPages{274}
\newlength{\thickCream}\setlength{\thickCream}{0.0025 in}
\newlength{\thickWhite}\setlength{\thickWhite}{0.002252 in}
\newlength{\SpineW}\setlength{\SpineW}{\numPages\thickWhite} % 236*0.0025 = 0.59 (cream),  262*0.002252 = 0.590024 (white)
% The standard bleed
\newlength{\Bleed}\setlength{\Bleed}{.125 in}
% Set the page size
\newlength{\PageW}\setlength{\PageW}{7.5 in}
\newlength{\PageH}\setlength{\PageH}{9.25 in}
% Calculate the total dimensions
\newlength{\TrimW}\setlength{\TrimW}{\dimexpr 2\PageW + \SpineW \relax}
\newlength{\TrimH}\setlength{\TrimH}{\PageH}
\newlength{\CropW}\setlength{\CropW}{\dimexpr \TrimW + 2\Bleed \relax}
\newlength{\CropH}\setlength{\CropH}{\dimexpr \TrimH + 2\Bleed \relax}
\newlength{\LineW}\setlength{\LineW}{2\Bleed}
%=======================================================
% Some PDF specials
%=======================================================
\special{pdf: docinfo <<
/Author (Exo7)
/Title (Python au Lycée - Algorithmes et programmation)
/Keywords (Python, Lycée)
/Subject (manuel de programmation en Python)
>>}
\edef\pwpt{\the\numexpr \dimexpr0.996264009963\CropW\relax/65536 \relax} %paper width in PS points
\edef\phpt{\the\numexpr \dimexpr0.996264009963\CropH\relax/65536 \relax} %paper height in PS points
\expandafter\special\expandafter{pdf: put @thispage <<
  /MediaBox [0 0 \pwpt\space\phpt]
  /BleedBox [0 0 \pwpt\space\phpt]
  /CropBox [0 0 \pwpt\space\phpt]
  /TrimBox [0 0 \pwpt\space\phpt]>>}
%=======================================================
% for test
%=======================================================
\usepackage{blindtext}
\newif\ifshowborders\showbordersfalse % to display the security zone (Bleed)
%\showborderstrue

\tikzset{
  frame/.style 2 args={inner sep=0, outer sep=0, minimum width=#1, minimum height=#2, node contents=}
}

\begin{document}%
  \begin{tikzpicture}
    % Clip
    \clip (-.5*\CropW,-\Bleed) rectangle +(\CropW,\CropH);
    % Set nodes for anchors : Cover, Back, Front and Spine
    \path
      (0,-\Bleed) node[name=Cover,above,fill=background,frame={\CropW}{\CropH}]
      (-\SpineW/2,0) node[name=Back,above left,frame={\PageW}{\PageH}]
      (\SpineW/2,0)  node[name=Front,above right,frame={\PageW}{\PageH}]
      (0,0)  node[name=Spine,above,spine,frame={\SpineW}{\PageH}];
    % ---------------------------------------------------
    % Background
    % ---------------------------------------------------

    % ---------------------------------------------------
    % Python logo and title
    % ---------------------------------------------------
    \begin{scope}[scale=0.7,transform canvas ={yshift=21mm}]
      % ---------------------------------------------------
      \path[every node/.style={scale=5, font=\titlefont, outer sep=0, inner sep=0}]
        (Front.center) +(21mm,21mm) node[titlepython]{PYTHON}
        (Front.center) +(-21mm,-21mm) node
          {\textcolor{titleau}{AU}\textcolor{titlelycee}{LYCÉE}};
      
      \path (Cover.east) +(55mm,-125mm) node[left,scale=1.8,text width=.40\PageW,align=left,fill=white,text=subtitlealgo,inner sep=.3em,font=\authorfont,rounded corners=10pt]{\bf ~~Algorithmes et programmation~~\vphantom{\Large t}};

      \path (Cover.east) +(55mm,-145mm) node[left,scale=1.8,text width=.20\PageW,align=left,fill=white,text=subtitlealgo,inner sep=.4em,font=\authorfont,rounded corners=10pt]{\bf ~~tome 1~~\vphantom{.}};
      
    \end{scope}
    % ---------------------------------------------------
    % Auteur
    % ---------------------------------------------------
    \path (Front.north) +(0,-21mm) node[author,text=subtitlealgo,scale=2,font=\authorfont]{\bf Arnaud Bodin -- François Recher};
    % ---------------------------------------------------
    % Spine
    % ---------------------------------------------------
    \begin{scope}[transform canvas ={yshift=21mm}]
      \node[rotate=90,scale=1.8] (tranche) at (Spine.center)
        {\textcolor{spinepython}{PYTHON}\textcolor{spineau}{AU}\textcolor{spinelycee}{LYCÉE}\textcolor{spinealgo}{~~~\textemdash~~tome 1}\vphantom{\large y}};
    \end{scope}
    % ---------------------------------------------------
    %  Logos Exo7
    % ---------------------------------------------------
    \path (Spine.south) +(0,14mm) node{\includegraphics[width=11mm]{exo7_logo_white}};
    \path (Front.south) +(0,14mm) node{\includegraphics[width=19mm]{exo7_logo_white}};
    % ---------------------------------------------------
    % Back
    % ---------------------------------------------------
    \path (Back.center) node[scale=1.6,text width=10.5cm,fill=white,fill opacity=.5,text opacity=.77,align = center]{
  \vspace*{2ex}    
 
  \footnotesize
  
  \hyphenchar\font=-1  % pas de césure
  
 
    
 \emph{Python} est le langage idéal pour apprendre la programmation.
C'est un langage puissant qui vous plongera dans le monde des algorithmes. 

\bigskip

Ce livre vous guide pas à pas à travers des activités mathématiques et informatiques originales adaptées au lycée. Il se complète par des ressources en ligne : tous les codes \emph{Python}, des vidéos et des fiches en couleurs. 
Vous avez tout en main pour réussir !   

\bigskip
\scriptsize

\begin{center}
\begin{minipage}{0.49\textwidth}
\begin{itemize}
\setlength{\itemsep}{-1pt}%
\item Premiers pas
\item Tortue (Scratch avec Python)
\item Si ... alors ...
\item Fonctions
\item Arithmétique -- Boucle tant que -- I
\item Chaînes de caractères% - Analyse d'un texte
\item Listes I
\item Statistique -- Visualisation de données
\item Fichiers
\item Arithmétique -- Boucle tant que -- II
\item Binaire I
\item Listes II
\item Binaire II
\end{itemize}
\end{minipage}
\begin{minipage}{0.49\textwidth}
\begin{itemize}
\setlength{\itemsep}{0pt}%
\item Probabilités -- Paradoxe de Parrondo
\item Chercher et remplacer
\item Calculatrice polonaise -- Piles
\item Visualiseur de texte -- Markdown
\item L-système
\item Images dynamiques
\item Jeu de la vie
\item Graphes et combinatoire de Ramsey
\item Bitcoin
\item Constructions aléatoires
\end{itemize}
\end{minipage}
\end{center}    

 \vspace*{3ex}   
    
    };
    % ---------------------------------------------------
    \ifshowborders
      % --- security zone
      \begin{scope}[line width=\LineW,red,opacity=.35]
        \draw (-.5*\TrimW,0) rectangle +(\TrimW,\TrimH);
        \draw (-.5*\SpineW-\Bleed/3,0) rectangle +(\SpineW+2\Bleed/3,\TrimH);
      \end{scope}
      % --- guides
      \begin{scope}[dashed]
        \draw (-.5*\TrimW,0) rectangle +(\TrimW,\TrimH);
        \draw (-.5*\SpineW,0) rectangle +(\SpineW,\TrimH);
      \end{scope}
    \fi
  \end{tikzpicture}%
\end{document}
